We have now introduced the main principles of Bayesian inference. Now, we can
turn to the Kalman filter, a popular and widely used algorithm for state 
estimation. The Kalman filter (KF, for short) is a recursive algorithm that uses
Bayesian inference to estimate the state of a dynamic system based on a sequence
of noisy measurements.

The key feature of the KF is the ability to predict future states of the system.
This is essential for applications that require to foresee the behavior of
tracked objects before the state of these objects is measured. Examples of such
systems may include autonomous driving or air defence systems. The prediction 
ability helps also to overcome situations when a sensor fails to measure the
position of an object (a so-called misdetection).

In the following subsections, we will introduce the state-space models, give
a formal definition of a general Bayes filter, infer the Kalman filter formulas
and discuss the Constant Velocity (CV) model, a common motion model used in the
Kalman filter.
