The Kalman filter is one of the most well-known and widely used algorithms in 
signal processing and control theory. It is a recursive algorithm that allows 
the estimation of internal states of entities in dynamic systems from a set of 
measurements that may be noisy or missing at some time steps. The filter was 
proposed by Rudolf Kalman in 1960 \cite{kalmanNewApproachLinear1960} and has 
since been pervasively used to control a vast array of consumer, health, 
commercial, and defense products \cite{grewalApplicationsKalmanFiltering2010}.

The development of the filter was motivated by the need to improve aerospace 
technology in the United States during the Cold War between the Soviet Block 
and the North American Treaty Organization. Because the Soviet Union managed to 
launch its artificial satellites and successfully send a human to space, the 
federal government of the United States supported research into new 
technologies in the aerospace area.

The Kalman filter is an example of a general Bayes filter that was introduced 
earlier. This means that the filter estimates the posterior distribution of the 
internal state at discrete time steps using prior information about the 
observed object's state and a set of noisy measurements. As briefly mentioned 
in Section \ref{sec:state-space}, the Kalman filter works on state-space linear 
models with Gaussian noise and a Gaussian prior of the state. The predict-
update loop in the Kalman filter is the same as in the Bayes filter, and the 
same Chapman-Kolmogorov equation defined in \ref{theorem:chapman-kolmogorov} 
and the update equation defined in \ref{theorem:bayes-filter-update} are 
incorporated.

The main advantage of the Kalman filter is that it allows for the estimation of 
the state of a system in real-time. It handles noisy measurements well and is 
fast; however, it is sensitive to initial parameter settings, such as the noise 
covariance matrices \cite{gePerformanceAnalysisKalman2016}. Nonetheless, there 
are new methods being developed that propose mechanisms to overcome this 
drawback, as in \cite{matiskoNoiseCovariancesEstimation2010} and 
\cite{yuenOnlineEstimationNoise2013}.

One of the main drawbacks of the Kalman filter is its Gaussian-linear 
assumption. In real-world applications, many systems exhibit non-linearity, and 
the Kalman filter may be ineffective. Nonetheless, several extensions of the 
filter have been proposed that address this. Two well-known algorithms are the 
Unscented Kalman Filter (UKF) and the Extended Kalman filter (EKF). The UKF is 
an algorithm that uses a set of carefully chosen sigma points to capture the 
true mean and covariance of the predicted and updated distributions without 
the need for linearization \cite{wanUnscentedKalmanFilter2000}. The EKF, on 
the other hand, linearizes the nonlinear motion and measurement models using a 
first-order Taylor expansion \cite{smithApplicationStatisticalFilter1962}. 
These methods have been proven to be effective in many applications, including 
the PHD filter. However, they are beyond the scope of this thesis and will not 
be covered in detail.
