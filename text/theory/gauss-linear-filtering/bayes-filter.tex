We have learned how the internal state of a system can be represented in terms
of a state vector and motion and measurement models. Now, we are ready to 
present the formal definition of the Bayes filter, the general abstraction of
any Bayesian filter, including the Kalman filter and the PHD filter.

First, we need to make a very important assumption on states. This assumption
is called the \textit{Markov model property}. This property states that, for 
the motion model, the present state of a system $x_k$ is dependent only on 
the state on the previous time step $x_{k-1}$ given all past states $x_{1:k-1}$ 
and measurements $z_{1:k-1}$. More formally:

\begin{equation}
p(x_k | x_1, \ldots, x_{k-2}, x_{k-1},
        z_1, \ldots, z_{k-2}, z_{k-1}) 
    = p(x_k | x_{k-1}).
\end{equation}

A similar requirement must hold for measurement model, that is:

\begin{equation}
    p(z_k | x_1, \ldots, x_{k-2}, x_{k-1}, x_{k}
        z_1, \ldots, z_{k-2}, z_{k-1}) 
    = p(z_k | x_k).
\end{equation}

The Markov property may seem restrictive but in reality it is not because state-
space models allows us to capture dependencies in the system dynamics by simply 
introducing more variables into the state vector. For instance, if we have a 
system where the current state has a dependence on the previous two states, we 
can augment the state vector to include the last two states as variables, and 
the Markov property will still hold.

The Bayes filter uses the motion and measurement models and assumes that this
property holds. The Bayes filter is a recursive framework that estimates an
internal state of the system over time using measurements. Every iteration of
the filter consists of two steps: prediction and update. On the prediction step,
the filter estimates a internal state $x_k$ based on the previous state
$x_{k-1}$ and the motion model $p(x_k | x_{k-1})$. The prediction step
is also known as the Chapman-Kolmogorov equation.

\begin{theorem}[The prediction step. The Chapman-Kolmogorov equation]\label{theorem:bayes-filter-predict}\label{theorem:chapman-kolmogorov}
    Given the set of measurements $z_{1:k-1} = \{z_1, z_2, \ldots, z_{k-1}\}$ 
    and the current state $x_{k-1}$ and the motion model $p(x_k | x_{k-1})$, 
    the prediction step of the Bayes filter is computed as follows:

    \begin{equation}
        p\left({x}_k | {z}_{1: k-1}\right)
        = \int 
            p\left(
                {x}_k, {x}_{k-1} | {z}_{1: k-1}
            \right)
            \mathrm{d} {x}_{k-1}
        = \int
            p\left(
                {x}_k | {x}_{k-1}\right
            ) p\left(
                {x}_{k-1} | {z}_{1: k-1}
            \right)
            \mathrm{d} {x}_{k-1},
    \end{equation}
    
    \noindent where the integral is taken over the entire state space of ${x}_{k-1}$, and $p\left({x}_{k-1} | {z}_{1: k-1}\right)$ is the posterior density of the state at time $k-1$, given all measurements up to time $k-1$.
\end{theorem}

Next, on the update step, the filter corrects the predicted state with the 
measurement $z_k$ using the measurement model $p(z_k | x_k)$. This step is
computed using the standard Bayes' rule.

\begin{theorem}[The update step]\label{theorem:bayes-filter-update}
    Given the output of the prediction step of the Bayes filter 
    $p\left({x}_k | {z}_{1: k-1}\right)$, the observed measurement $z_k$
    and the measurement model $p(z_k | x_k)$, the update step of the Bayes
    filter is computed as follows:
    
    \begin{equation}
        p\left({x}_k | {z}_{1: k}\right)=\frac{p\left({z}_k | {x}_k\right) p\left({x}_k | {z}_{1: k-1}\right)}{p\left({z}_k | {z}_{1: k-1}\right)} \propto p\left({z}_k | {x}_k\right) p\left({x}_k | {z}_{1: k-1}\right).
    \end{equation}
\end{theorem}

These two steps create a loop, and to use the filter, we first predict the next state using the Chapman-Kolmogorov equation, and then we update our guess with a measurement. Note, that we will often use the simplified notation to explicitly state the time step of the value. The notation $\vecat{\bullet}{k|k-1}$ represents the predicted value at time step $k$, and the notation $\vecat{\bullet}{k|k}$ represents the updated value at time step $k$ after incorporating a measurement.

Now, having introduced the general Bayes filter, we will continue with 
exploring the Kalman filter in detail, the popular and widely used tool for 
state estimation.
