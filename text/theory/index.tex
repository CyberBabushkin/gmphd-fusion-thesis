The field of target tracking, both single and multiple, rely on mathematical concepts and methods based on probability theory. In this chapter, we provide an introduction to key concepts of probability and inference. We start by reviewing the basics of probability theory, including random variables, probability distributions with examples of the most important ones for multi-target tracking, and the Bayes' theorem. Then, we describe the concept of state-space models, a powerful framework that allows to describe dynamic system evolving in time. Then, we apply these concepts with a Bayes filter, a general method for recursive estimation the state of a dynamic system using measurements obtained from a sensor over time. Finally, we define and prove two important in tracking theorems, and use these theorems to derive the Kalman filter, and important and widely used framework that provides an optimal state estimate of a linear system in the presence of Gaussian noise. The theoretical foundations in this chapter serve as a basis for the next chapter, which covers more sophisticated algorithms for tracking multiple targets in noisy environments.
