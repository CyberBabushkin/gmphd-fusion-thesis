In this section, we will briefly cover one of challenges in multi-target tracking, which is track maintenance and then we introduce the way how this challenge may be addressed in the GM-PHD filter. When an algorithm creates estimates of targets at each time step, these estimates cannot always be implicitly connected to those from previous time steps. If we require not only point estimates but also the trajectories of targets, we should use techniques to maintain identities of targets and to create associations between new measurements and these identities. Thus, the trajectory here refers to the collection of state estimates of a target in time. Generally, the track maintenance problem includes three disjoint sub-problems: track initialization, track confirmation and track deletion \cite{blackmanDesignAnalysisModern1999}. Track initialization refers to the creation of new tracks. A new track is initialized when a new measurement is received that cannot be assigned to any existing track. Since the measurement is not always the true measurement, the tentative track should be confirmed by receiving subsequent measurements that meet certain criteria. Different algorithm specify various different criteria depending on their internal logic. For instance, one way to confirm tracks is called the $M/N$ rule, which states that the track becomes confirmed if at least $M$ measurements out of $N$ subsequent updates should be assigned to the tentative track \cite[871]{blackmanDesignAnalysisModern1999}. Confirmed tracks create trajectories of objects and we need a way to terminate tracks that do not receive measurements anymore, for example, when a target leaves the field of view. Yet again, there are multiple different ways how to approach this problem, including restricting the maximum age of a track, or when the likelihood of target existence falls under some threshold \cite[243]{challaFundamentalsObjectTracking2011}.

Estimating tracks has several challenges that need to be addressed to implement an effective tracking algorithm. For example, the track loss occurs when a track is temporarily or permanently lost due to a missed detection or incorrect association between a measurement and some existing track. Wrong associations may also cause track swaps, when an algorithm swaps measurements from two or more targets moving closely to each other, and measurements generated by one target are assigned to a track of another, and vice versa. Closely moving targets may also create a so-called track coalescence, the situation when two tracks of different targets are merged into one track in the middle of trajectories of these two targets. For instance, the JPDA filter is known to suffer from track coalescence \cite{kropfreiterTrackCoalescenceRepulsion2021}.

Despite there are many advanced techniques how to maintain track, we choose a straightforward approach in this work for the implementation of the GM-PHD filter. Every Gaussian component receives a unique tag, or label, that identifies this component. When the algorithm generates state estimates, these tags are used to connect single-target estimates from different time steps and having the same label into one trajectory. The track is terminated when the Gaussian component with the tag assigned to this track is removed from the mixture. While this approach is intuitive and simple, in Chapter \ref{ch:results} we show that is shows good performance results.
