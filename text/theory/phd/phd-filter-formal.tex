We have defined the PHD function and mentioned that instead of propagating the full posterior density, the PHD filter propagates the first-order moment of the multi-target posterior state. In this section, we will describe the main assumptions of the PHD filter and derive the predict and update step equations in terms of the PHD function. It should be noted that while the PHD filter is a general concept, its implementation is specific to the problem domain. Later sections will provide the exact closed-form solution for Gaussian-linear problems.

The PHD filter relies on several general assumptions about the modeled problem. In addition to the standard assumptions, the PHD filter assumes:

\begin{enumerate}
\item Each target has a single-target motion model $f_k(\vecat{x}{k} | \vecat{x}{k-1})$.
\item Existing targets move to the next state with probability $P_{S,k}(\vecat{x}{k-1})$.
\item New targets are born with states from $X$ according to the multi-target birth likelihood $\gamma_k(X)$. The corresponding PHD of this likelihood is:
\begin{equation}
\gamma_k(\mathbf{x}) = \int \gamma_k({\mathbf{x}} \cup X) \delta X.
\end{equation}
\item Existing targets generate measurements according to the single-target Markovian measurement model $g(\vecat{z}{k} | \vecat{x}{k})$ with probability $P_{D,k}(\vecat{x}{k})$.
\item Clutter measurements are Poisson-distributed in time with the spatial distribution $\kappa_k(\vecat{z}{k})$.
\item The predicted multi-target distribution $p_{k|k-1}(\vecat{x}{k} | \vecat{z}{1:k-1})$ is a Poisson RFS.
\end{enumerate}

The last assumption is a simplification that allows to derive a closed form solution. A Poisson process assume that all targets are statistically independent, and such approximation is not that restrictive is the interactions between moving targets in negligible. In this work we do not include the spawning process, therefore, it can be shown that this requirement is satisfied \cite{voGaussianMixtureProbability2006}.

Both the prediction and the update follow the same rules we have derived in Section \ref{sec:bayes-filter-rfs}. It can be shown that the Chapman-Kolmogorov equation and the measurement correction step for the PHD filter, excluding target spawning, has the following form:

\begin{theorem}[The PHD recursion]\label{theorem:phd-recursion}
    Assuming that all requirements defined above hold, the PHD recursion has the following form:

    \begin{align}
        v_{k|k-1}(\vecat{x}{k} | Z_{1:k-1})
        &= \gamma_k(\vecat{x}{k}) + \int P_{S,k}(\vecat{x}{k-1}) f_{k|k-1}(\vecat{x}{k} | \vecat{x}{k-1}) v_{k-1}(\vecat{x}{k-1} | Z_{1:k-1}) \mathrm{d} \vecat{x}{k-1}, \\
        v_k(\vecat{x}{k} | Z_{1:k})
        &= \left[1-P_{D,k}(\vecat{x}{k})\right] v_{k|k-1}(\vecat{x}{k} | Z_{1:k-1}) \nonumber \\
        &+ \sum_{\vecat{z}{k} \in Z_k} \frac{
            P_{D,k}(\vecat{x}{k}) g_k(\vecat{z}{k} | \vecat{x}{k}) v_{k|k-1}(\vecat{x}{k} | Z_{1:k-1})
        }{
            \kappa_k(\vecat{z}{k}) + \int P_{D,k}(\boldsymbol{\xi}) g_k(\vecat{z}{k} | \boldsymbol{\xi}) v_{k|k-1}(\boldsymbol{\xi} | Z_{1:k-1}) \mathrm{d} \boldsymbol{\xi}
        },
    \end{align}

    \noindent where $v_{k-1}(\vecat{x}{k-1} | Z_{1:k-1})$ is the posterior PHD function from the previous time step \cite{voGaussianMixtureProbability2006} \cite[588--591]{mahlerStatisticalMultisourcemultitargetInformation2007}.
\end{theorem}

Note that the computation of the posterior probability hypothesis density after the update does not include the set integral, which means that we do not need to compute all association hypotheses between measurements and states, decreasing the computational burden. It can be shown that the computational complexity of the PHD filter at every time step is $\mathcal{O}(mn)$, where $m$ is the number of measurements and $n$ is the number of current targets \cite[592]{mahlerStatisticalMultisourcemultitargetInformation2007}.

In the next section, we will provide the formulation of the PHD filter in terms of Gaussian-linear filtering and Gaussian mixtures. We will see that for such cases a closed-form solution exists, and there are no approximations required to compute the posterior PHD density.
