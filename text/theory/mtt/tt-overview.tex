There are multiple filters that use association hypotheses. For single-target tracking, when the maximum number of targets is fixed to one, in Gaussian-linear scenarios the Probabilistic Data Association (PDA) filter can be utilized \cite{bar-shalomProbabilisticDataAssociation2009}. At each time step, this filter creates new hypotheses for all possible data associations and creates a joint posterior distribution after at the update step. Each hypothesis is assigned with a weight that reflects the likelihood that it actually originates from the target. Next, the resulting Gaussian mixture is reduced to only one Gaussian that represents the posterior state of the target. Both Gaussian mixtures and the reduction techniques will will be discussed later in this work.

This PDA filter is conceptually very simple and straightforward, but it may be too simple for many real-world scenarios, particularly when there are multiple targets present that are too close to each other to be tracked by multiple instance of the PDA filter. This is addressed in the extension and the resulting filter is named the Joint Probabilistic Data Association (JPDA) filter \cite{bar-shalomMultitargetmultisensorTrackingPrinciples1995}. It can handle multiple targets all at once but only when the number of targets is known in advance. The main idea behind this filter is that at each time step it creates all possible measurement-to-target association hypotheses, and, for each target, it creates a joint probability from all partial probabilities computed for each measurement. The series of such probabilities create tracks, and the track with the highest probability is considered the real track of the object. Because of the way how hypotheses are calculated, the JPDA filter is computationally much more expensive than the PDA filter. Moreover, if tracks get too close to each other, this filter shows the problem called the track coalescence—when tracks of two targets merge into one track.

Both PDA and JPDA filters are single-scan methods, which means they process only one set of measurements at a time. Compared to single-scan methods, there exist multi-scan methods that compute probabilities of tracks based on the history of all measurements, in a hierarchical manner. This way of calculating posterior probabilities leads to significant improvements in tracking accuracy, since the filter takes into consideration the whole history of the object movement. The classical example of a multi-scan filter is Reid's Multiple Hypothesis Tracker (MHT), also known as the hypothesis-oriented MHT (HO-MHT) \cite{reidAlgorithmTrackingMultiple1979}. This filter is similar to the way it computes probabilities; however, the calculation of the probability of each association hypothesis incorporates the probability of the parent hypothesis from the previous time step, thus creating a hypothesis tree at each filter cycle. The number of hypotheses is therefore multiplied by the number of new measurements at each step, and the total number of association hypotheses grows exponentially. Efficient implementations of the recursive HO-MHT include advanced techniques on how to prune the number of less probable hypotheses at each time step \cite{coxEfficientImplementationReid1996}.

The computational complexity of Reid's MHT has led to modifications in the way hypotheses are created. Instead of generating new hypotheses for all parent hypotheses at each time step, we can create several sequences of the best associations for several time steps in the past and compute new hypothesis probabilities for those hypotheses only. This reduction in the computational complexity avoids the need to compute all possible branches in the hypothesis tree. Moreover, it allows for efficient implementation techniques like look-up tables, and there is no recursion in the computation of new hypotheses. This algorithm is known as the track-oriented MHT (TO-MHT) \cite{werthmannStepbystepDescriptionComputationally1992}. As a modification of Reid's MHT, TO-MHT is also a multi-scan method, but instead of recursively evaluating all possible hypotheses, it processes all hypotheses at once.

The way both variants of MHT handle track hypothesis initialization and the propagation of association hypotheses over time allows ``the MHT approach inherently handle initiation and termination of tracks, and hence accommodate an unknown and time-varying number of targets'' \cite{voMultitargetTracking2015}. However, because the number of hypotheses grows exponentially, hypothesis reduction techniques should be used, and the pruning of hypotheses with low probabilities should be relatively aggressive. This makes MHT a strong algorithm but with higher computational requirements.

The JPDA filter and the MHT are two approaches for handling multiple targets in a scene. However, there is one alternative approach that is conceptually very different from both methods, which utilizes an abstract mathematical concept called Finite-Set Statistics (FISST) \cite{mahlerStatisticalMultisourcemultitargetInformation2007}. We will dedicate several next sections to explaining what FISST is, introducing the main theoretical assumptions of the multiple target tracking problem, and then discussing the main building blocks of the PHD filter.
