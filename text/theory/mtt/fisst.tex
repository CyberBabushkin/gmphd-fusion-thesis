We have discussed one possible approach for evaluating the uncertainty between targets and measurements in multi-target tracking. Filters such as JPDA or MHT, in both variants, use a hypothesis-based approach where at each time step, a set of hypotheses is created to map all possible associations between new measurements and existing targets. However, these approaches do not model the uncertainty of the number of targets themselves. At any given moment, new targets may appear or disappear from the field of view, and the filters should be able to estimate tracks for every target.

To illustrate this point, let us consider the scenario of a common surveillance camera tracking people in a large mall. At every moment, people enter the area covered by the camera, cross it, and then leave the area. The number of people can vary greatly at different times of the day or on different days of the week, and there is no simple solution for estimating the number of mall visitors at every moment.

In the JPDA filter, the number of hypotheses is assumed to be known in advance, which is a rare case in real-world applications. In the MHT approach, object appearance events can be modeled by assuming that if two or more measurements have a high probability of being assigned to one target, this is probably because the number of objects in the vicinity of the target is greater than one. However, these approaches are not systematic nor optimal, and the exact estimation of the number of targets is rather a side effect \cite{challaFundamentalsObjectTracking2011}.

This problem led to the development of the Random Finite Sets (RFSs) theory, a completely new approach to multi-target tracking. At each time step, a collection of objects that are present in the field of view of a sensor is modeled as a random finite set. Generally, a set is a collection of distinct objects without a specific order of the elements. In an RFS, the number of objects is only known to be finite, but the exact cardinality, or the number of elements in the set, is modeled using some probability distribution. Moreover, all elements in the set are probability distributions. Therefore, RFSs are random variables that model the uncertainty of every state and the number of states as a single entity. An outcome, or a realization, of such a random variable is a fixed set with an exact state, and measurements are considered to be outcomes of the unknown internal state.

The RFS concept led to the development of a new branch in statistics called Finite Sets Statistics (FISST). In FISST, sets are random variables that have their probability distributions and probability density functions. Furthermore, FISST allows the use of RFSs for Bayesian inference, which leads to the need for defining basic algebraic operations on sets, such as integrals and derivatives. As we will soon see, the mathematics behind it becomes complex, and one way to simplify the notation is the introduction of a new way of expressing relations. This concept, called probability generating functionals (p.g.fls), significantly simplifies the notation in FISST. Unfortunately, the underlying level of mathematical abstraction becomes much more difficult to understand.

In this section, we will give a formal definition for a random finite set and also define the main formulas that are needed to understand the logic behind FISST and use it to infer the PHD filter. In this work, we will use standard notation that does not use p.g.fls. However, the reader may refer to the original work in \cite{mahlerMultitargetBayesFiltering2003} to learn about these concepts.
