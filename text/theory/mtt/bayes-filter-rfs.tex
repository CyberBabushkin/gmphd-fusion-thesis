In Section \ref{sec:bayes-filter}, we have discussed the prediction step, also known as the Chapman-Kolmogorov equation, and the update step in term of state vectors and motion and measurement models. In this section, we will derive similar formulas for Bayesian inference in the context of random finite sets.

Since RFS pdfs are indeed densities, as we discussed in Section \ref{sec:rfs-definition}, the recursive relations for RFS have the same form, however, densities are multi-target densities that catch the motion of all objects at once, and arguments are sets.

\begin{theorem}[The prediction step for RFS]{theorem:bayes-filter-predict-rfs}
    Given sets of measurements $Z_{1:k-1} = \{Z_1,\allowbreak Z_2, \ldots, Z_{k-1}\}$, the current state RFS $X_{k-1}$, the multi-target transition density $f_{k|k-1}(X_k | X_{k-1})$, and the multi-target posterior density from the previous time step $p_{k-1}\left({X}_{k-1} | {Z}_{1:k-1}\right)$, the prediction step of the Bayes filter for RFSs is computed as follows:

    \begin{equation}
        p_{k|k-1}\left({X}_k | {Z}_{1:k-1}\right)
        = \int
            f_{k|k-1}\left( {X}_k | {X}_{k-1} \right)
            p_{k-1}\left( {X}_{k-1} | {Z}_{1: k-1} \right)
            \delta {X}_{k-1}.
    \end{equation}
\end{theorem}

\begin{theorem}[The update step for RFS]\label{theorem:bayes-filter-update-rfs}
    Given the predicted density $p_{k|k-1}\left({X}_k | {Z}_{1:k-1}\right)$, the observed set of measurements $Z_k$ and the multi-target likelihood $g_k(Z_k | X_k)$, the update step of the Bayes filter for RFS is computed as follows:
    
    \begin{align}
        p_{k}\left({X}_k | {Z}_{1: k}\right)
        &= \frac{
            g_k\left({Z}_k | {X}_k\right) p_{k|k-1}\left({X}_k | {Z}_{1: k-1}\right)
        }{
            \int g_k\left({Z}_k | X\right) p_{k|k-1}\left(X | {Z}_{1: k-1}\right) \delta X
        } \\
        &\propto g_k\left({Z}_k | {X}_k\right) p_{k|k-1}\left({X}_k | {Z}_{1: k-1}\right).
    \end{align}
\end{theorem}

Note that the multi-target transition density is not the same motion model that we used before. The same applies for the multi-target likelihood that plays the role of the measurement model but is a completely different relation. The transition density captures the motion of all objects in a set all at once, including the birth (appearance) of new objects, and the death (disappearance). The likelihood, on the other hand, models the transition from a set of object states to probable measurements. Arguments of both are now sets, and explicit forms can be derived from motion and measurement models using FISST. These derivations along with formal proofs of the Bayesian recursion defined above can be found in \cite{mahlerMultitargetBayesFiltering2003}.

Even though the relations look similarly on the first glance, the underlying mathematics differs drastically. In comparison to the single-object tracking, states now represent the distribution of all objects simultaneously. Moreover, the multi-target transition density now captures much more information: not only it describes the transition of objects that are already in the current state set, but also the behavior of the change of the cardinality of the set. In the next section, we will explore how these relations can be applied to address the multi-target tracking problem. We will also examine the assumptions that must be made in order to develop Bayes filters based on FISST.
