The Gaussian distribution, also known as the normal distribution, is a 
continuous probability distribution that is widely used in many branches
of statistics, including Bayesian filtering. It is typically used to model
a large number of independent and identically distributed (i.i.d.) random variables.

The Gaussian distribution is characterized by two parameters: the mean 
$\mu$ and the variance $\sigma^2$. Its probability density function (pdf), 
expected value, and variance are given by:

\begin{align}
    p(x)
        =\mathscr{N}\left(x ; \mu, \sigma^2\right)
        &=\frac{1}{\sqrt{2 \pi} \sigma} \exp \left(\frac{(x-\mu)^2}{2 \sigma^2}\right), \\
    E[X] &= \mu, \\
    \operatorname{Var}[X] &= \sigma^2.
\end{align}

In the notation $\mathscr{N}\left(x ; \mu, \sigma^2\right)$, the first 
parameter $x$ means ``evaluated at.''
Despite the cdf of the Gaussian distribution not having a closed-form 
representation, the distribution possesses several desirable properties that 
allow for closed-form solutions in many applications, including object 
tracking. Although it may be too simple to model certain scenarios, it 
represents the approximate average position of objects and their uncertainty 
quite well. This work utilizes mixtures of Gaussians, which will be discussed 
in more detail later.
