% - State-space model
% - Bayes filter
% - Kalman filter
% - Constant velocity model (incl. the measurement model, which is a part of CVM)

We have now introduced the main principles of Bayesian inference. Now, we can
turn to the Kalman filter, a popular and widely used algorithm for state 
estimation. The Kalman filter (KF, for short) is a recursive algorithm that uses
Bayesian inference to estimate the state of a dynamic system based on a sequence
of noisy measurements.

The key feature of the KF is the ability to predict future states of the system.
This is essential for applications that require to foresee the behavior of
tracked objects before the state of these objects is measured. Examples of such
systems may include autonomous driving or air defence systems. The prediction 
ability helps also to overcome situations when a sensor fails to measure the
position of an object (a so-called misdetection).

In the following subsections, we will introduce the state-space models, give
a formal definition of a general Bayes filter, infer the Kalman filter formulas
and discuss the Constant Velocity (CV) model, a common motion model used in the
Kalman filter.

\subsection{State-space model}\label{sec:state-model}

In the previous section, we established the relationships between prior and 
posterior distributions and learned about recursive Bayesian estimation of the 
posterior. Now, we will explore these concepts in the context of object 
tracking.

Object tracking involves estimating the internal states of objects that move in 
some space, which are often unknown to us. These states may include physical 
properties of objects such as position, velocity, acceleration, and 
orientation, depending on the application. Sensors such as cameras, infrared 
scanners, lidars or radars constantly generate measurements of the object 
states, which may be the distance of an object to the sensor, or its 
temperature. However, these measurements are often noisy due to environmental 
conditions or imperfections of the sensor. We use Bayesian filtering to 
estimate the real state of the objects and filter out the noise.

To handle the complexity and diversity of possible situations and properties, 
we need a systematic framework for modeling object motion that captures all 
sources of uncertainty and the nature of physical behavior of objects. This 
framework is called state-space modeling. In state-space models, the internal 
state we want to estimate is represented as a vector of variables that evolve 
over time according to a set of rules specified by the \textit{motion model}, 
which is a known function. The measurements obtained at each time step are 
modeled as a different function of the state variables, known as the 
\textit{measurement model}\footnote{
    It should be mentioned that the names `motion' and `measurement' models are 
    not the only terms used to describe them. The reader may also encounter 
    terms such as kinematic, state-transition, dynamic, or system models for the 
    motion model, and terms like observation, sensor, or likelihood models for 
    the measurement model. All of these names are correct, and their use depends 
    on the context in which they are applied. However, in this work, the author 
    strives for consistency and will use the terms `motion' and `measurement' 
    models throughout the entire text.
}.

For example, consider a moving bicycle and a surveillance camera with a 
rectangular field of view (FoV). The bicycle enters the FoV and crosses it at a 
constant speed. The camera has an algorithm that estimates the position of 
objects in the space from the image. With each frame $f$, it returns a new 
position, say $(\hat x_f, \hat y_f)$. We can represent the state of the bicycle 
as a vector of the position and velocity, which are unknown variables. Based on 
the measurements, we estimate their values using the known physical relation 
between distance, velocity, and time: $d = v \Delta t$, where $d$ is the 
distance that the object with velocity $v$ traveled in time $\Delta t$.

Both the motion and measurement models should include some noise. In the 
example above, the movement of the bicycle cannot perfectly follow the formula, 
and there will always be some minor changes in speed. Moreover, the measurement 
model, which translates an estimated state into a measurement, will also 
contain some uncertainty since the camera does not have a perfect 
representation of the space.

We can now describe these models formally. Note that since the state is a 
vector, we will use $\mathbf{x}$ for vectors. The motion and measurement models 
can be generally expressed as:

$$
\begin{aligned}
    p(\vecat{x}{k} | \vecat{x}{k-1}) 
        &= f_t(\vecat{x}{k-1}, \vecat{u}{k}, \vecat{w}{k}), \\
    p(\vecat{z}{k} | \vecat{x}{k}) 
        &= h_t(\vecat{x}{k}, \vecat{v}{k}),
\end{aligned}
$$

where $f_t$ and $h_t$ are functions, $\vecat{x}{k-1}$ is the estimated state 
from time step $k-1$, $\vecat{u}{k}$ is called the input (or control) vector, 
and $\vecat{w}{k}$ and $\vecat{v}{k}$ are zero-mean noise random variables.

However, this definition is too general since the functions $f_t$ and $h_t$ can 
be anything. As mentioned earlier, to get a closed-form solution in the 
Bayesian inference framework, we need to choose conjugate distributions for the 
likelihood and the prior. The Kalman filter, which we will describe soon, works 
for the Gaussian-linear case, where the functions $f_t$ and $h_t$ are linear 
and noise variables are distributed as Gaussian, i.e., 
$\vecat{w}{k}, \vecat{v}{k} \sim \mathscr{N}(\mathbf{0}, \Sigma)$. The Gaussian-
linear state-space model has the following representation:

$$
\begin{aligned}
    p(\vecat{x}{k} | \vecat{x}{k-1}) 
        &= \mathbf{F} \vecat{x}{k-1}
            + \mathbf{B} \vecat{u}{k}
            + \vecat{w}{k},
        & \vecat{v}{k} \sim \mathscr{N}(\mathbf{0}, \mathbf{Q}), \\
    p(\vecat{z}{k} | \vecat{x}{k})
        &= \mathbf{H} \vecat{x}{k} + \vecat{v}{k},
        & \vecat{w}{k} \sim \mathscr{N}(\mathbf{0}, \mathbf{R}), \\
    p(\vecat{x}{0}) &\sim \mathscr{N}(\vecat{\hat{x}}{0}), \vecat{P}{0}),&
\end{aligned}
$$

where:

\begin{itemize}
    \item $\mathbf{F}$ is the \textit{transition matrix},
    \item $\mathbf{B}$ is the \textit{input matrix},
    \item $\mathbf{H}$ is the \textit{measurement matrix},
    \item $\mathbf{Q}$ and $\mathbf{R}$ are symmetric positive definite matrices
        that describe the statistical properties of the motion noise
        \vecat{v}{k} and the measurement noise \vecat{w}{k}, respectively,
    \item $\vecat{\hat x}{0}$ and $\vecat{P}{0}$ are mean and variance of the 
    prior state.
\end{itemize}

The control variable $\vecat{u}{k}$, in general, represents some input signal from the environment, and $\mathbf{B}$ specifies how the input signal affects the dynamic system. This may include the effect of gravity on the vertical traveled distance or the voltage applied to some circuit. However, in object tracking, we rely on measurements obtained from sensors to update our estimate of the object's state. We do not have control over the object's motion. Thus, the input matrix is often redundant, and we omit it (or set it to a zero matrix) along with the input variable $\vecat{u}{k}$.

The notation above may be inconvenient in some cases. In this work, we will often use a shorter version that has the same meaning:

$$
\begin{aligned}
    p(\vecat{x}{k} | \vecat{x}{k-1})
        &= \mathscr{N}(\vecat{x}{k}; \mathbf{F}\vecat{x}{k-1}, \mathbf{Q}), \\
    p(\vecat{z}{k} | \vecat{x}{k})
        &= \mathscr{N}(\vecat{z}{k}; \mathbf{H}\vecat{x}{k}, \mathbf{R}), \\
    \vecat{x}{0}
        &\sim \mathscr{N}(\vecat{x}{0}; \vecat{\hat{x}}{0}, \vecat{P}{0}).
\end{aligned}
$$

\subsection{The Bayes filter}\label{sec:bayes-filter}

We have learned how the internal state of a system can be represented in terms
of a state vector and motion and measurement models. Now, we are ready to 
present the formal definition of the Bayes filter, the general abstraction of
any Bayesian filter, including the Kalman filter and the PHD filter.

First, we need to make a very important assumption on states. This assumption
is called the \textit{Markov model property}. This property states that, for 
the motion model, the present state of a system $x_k$ is dependent only on 
the state on the previous time step $x_{k-1}$ given all past states $x_{1:k-1}$ 
and measurements $z_{1:k-1}$. More formally:

$$
p(x_k | x_1, \ldots, x_{k-2}, x_{k-1},
        z_1, \ldots, z_{k-2}, z_{k-1}) 
    = p(x_k | x_{k-1}).
$$

A similar requirement must hold for measurement model, that is:

$$
p(z_k | x_1, \ldots, x_{k-2}, x_{k-1}, x_{k}
        z_1, \ldots, z_{k-2}, z_{k-1}) 
    = p(z_k | x_k).
$$

The Markov property may seem restrictive but in reality it is not because state-
space models allows us to capture dependencies in the system dynamics by simply 
introducing more variables into the state vector. For instance, if we have a 
system where the current state has a dependence on the previous two states, we 
can augment the state vector to include the last two states as variables, and 
the Markov property will still hold.

The Bayes filter uses the motion and measurement models and assumes that this
property holds. The Bayes filter is a recursive framework that estimates an
internal state of the system over time using measurements. Every iteration of
the filter consists of two steps: prediction and update. On the prediction step,
the filter estimates a internal state $x_k$ based on the previous state
$x_{k-1}$ and the motion model $p(x_k | x_{k-1})$. The prediction step
is also known as the Chapman-Kolmogorov equation.

\begin{theorem}[The prediction step. The Chapman-Kolmogorov equation]\label{theorem:bayes-filter-predict}\label{theorem:chapman-kolmogorov}
    Given the set of measurements $z_{1:k-1} = \{z_1, z_2, \ldots, z_{k-1}\}$ 
    and the current state $x_{k-1}$ and the motion model $p(x_k | x_{k-1})$, 
    the prediction step of the Bayes filter is computed as follows:

    $$
    p\left({x}_k | {z}_{1: k-1}\right)
        = \int 
            p\left(
                {x}_k, {x}_{k-1} | {z}_{1: k-1}
            \right)
            \mathrm{d} {x}_{k-1}
        = \int
            p\left(
                {x}_k | {x}_{k-1}\right
            ) p\left(
                {x}_{k-1} | {z}_{1: k-1}
            \right)
            \mathrm{d} {x}_{k-1}.
    $$
\end{theorem}

Next, on the update step, the filter corrects the predicted state with the 
measurement $z_k$ using the measurement model $p(z_k | x_k)$. This step is
computed using the standard Bayes' rule.

\begin{theorem}[The update step]\label{theorem:bayes-filter-update}
    Given the output of the prediction step of the Bayes filter 
    $p\left({x}_k | {z}_{1: k-1}\right)$, the observed measurement $z_k$
    and the measurement model $p(z_k | x_k)$, the update step of the Bayes
    filter is computed as follows:
    
    $$
    p\left({x}_k | {z}_{1: k}\right)=\frac{p\left({z}_k | {x}_k\right) p\left({x}_k | {z}_{1: k-1}\right)}{p\left({z}_k | {z}_{1: k-1}\right)} \propto p\left({z}_k | {x}_k\right) p\left({x}_k | {z}_{1: k-1}\right).
    $$
\end{theorem}

Now, having introduced the general Bayes filter, we will continue with 
exploring the Kalman filter in detail, the popular and widely used tool for 
state estimation.

\subsection{The Kalman filter}

The Kalman filter is one of the most well-known and widely used algorithms in 
signal processing and control theory. It is a recursive algorithm that allows 
the estimation of internal states of entities in dynamic systems from a set of 
measurements that may be noisy or missing at some time steps. The filter was 
proposed by Rudolf Kalman in 1960 \cite{kalmanNewApproachLinear1960} and has 
since been pervasively used to control a vast array of consumer, health, 
commercial, and defense products \cite{grewalApplicationsKalmanFiltering2010}.

The development of the filter was motivated by the need to improve aerospace 
technology in the United States during the Cold War between the Soviet Block 
and the North American Treaty Organization. Because the Soviet Union managed to 
launch its artificial satellites and successfully send a human to space, the 
federal government of the United States supported research into new 
technologies in the aerospace area.

The Kalman filter is an example of a general Bayes filter that was introduced 
earlier. This means that the filter estimates the posterior distribution of the 
internal state at discrete time steps using prior information about the 
observed object's state and a set of noisy measurements. As briefly mentioned 
in Section \ref{sec:state-model}, the Kalman filter works on state-space linear 
models with Gaussian noise and a Gaussian prior of the state. The predict-
update loop in the Kalman filter is the same as in the Bayes filter, and the 
same Chapman-Kolmogorov equation defined in \ref{theorem:chapman-kolmogorov} 
and the update equation defined in \ref{theorem:bayes-filter-update} are 
incorporated.

The main advantage of the Kalman filter is that it allows for the estimation of 
the state of a system in real-time. It handles noisy measurements well and is 
fast; however, it is sensitive to initial parameter settings, such as the noise 
covariance matrices \cite{gePerformanceAnalysisKalman2016}. Nonetheless, there 
are new methods being developed that propose mechanisms to overcome this 
drawback, as in \cite{matiskoNoiseCovariancesEstimation2010} and 
\cite{yuenOnlineEstimationNoise2013}.

One of the main drawbacks of the Kalman filter is its Gaussian-linear 
assumption. In real-world applications, many systems exhibit non-linearity, and 
the Kalman filter may be ineffective. Nonetheless, several extensions of the 
filter have been proposed that address this. Two well-known algorithms are the 
Unscented Kalman Filter (UKF) \cite{wanUnscentedKalmanFilter2000a} and the 
Extended Kalman filter (EKF) \cite{smithApplicationStatisticalFilter1962}. The 
UKF is an algorithm that uses a set of carefully chosen sigma points to capture 
the true mean and covariance of the predicted and updated distributions without 
the need for linearization. The EKF, on the other hand, linearizes the 
nonlinear motion and measurement models using a first-order Taylor expansion. 
These methods have been proven to be effective in many applications, including 
the PHD filter. However, they are beyond the scope of this thesis and will not 
be covered in detail.

