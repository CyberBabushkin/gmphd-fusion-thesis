
In the previous section, we established the relationships between prior and 
posterior distributions and learned about recursive Bayesian estimation of the 
posterior. Now, we will explore these concepts in the context of object 
tracking.

Object tracking involves estimating the internal states of objects that move in 
some space, which are often unknown to us. These states may include physical 
properties of objects such as position, velocity, acceleration, and 
orientation, depending on the application. Sensors such as cameras, infrared 
scanners, lidars or radars constantly generate measurements of the object 
states, which may be the distance of an object to the sensor, or its 
temperature. However, these measurements are often noisy due to environmental 
conditions or imperfections of the sensor. We use Bayesian filtering to 
estimate the real state of the objects and filter out the noise.

To handle the complexity and diversity of possible situations and properties, 
we need a systematic framework for modeling object motion that captures all 
sources of uncertainty and the nature of physical behavior of objects. This 
framework is called state-space modeling. In state-space models, the internal 
state we want to estimate is represented as a vector of variables that evolve 
over time according to a set of rules specified by the \textit{motion model}, 
which is a known function. The measurements obtained at each time step are 
modeled as a different function of the state variables, known as the 
\textit{measurement model}\footnote{
    It should be mentioned that the names `motion' and `measurement' models are 
    not the only terms used to describe them. The reader may also encounter 
    terms such as kinematic, state-transition, dynamic, or system models for the 
    motion model, and terms like observation, sensor, or likelihood models for 
    the measurement model. All of these names are correct, and their use depends 
    on the context in which they are applied. However, in this work, the author 
    strives for consistency and will use the terms `motion' and `measurement' 
    models throughout the entire text.
}.

For example, consider a moving bicycle and a surveillance camera with a 
rectangular field of view (FoV). The bicycle enters the FoV and crosses it at a 
constant speed. The camera has an algorithm that estimates the position of 
objects in the space from the image. With each frame $f$, it returns a new 
position, say $(\hat x_f, \hat y_f)$. We can represent the state of the bicycle 
as a vector of the position and velocity, which are unknown variables. Based on 
the measurements, we estimate their values using the known physical relation 
between distance, velocity, and time: $d = v \Delta t$, where $d$ is the 
distance that the object with velocity $v$ traveled in time $\Delta t$.

Both the motion and measurement models should include some noise. In the 
example above, the movement of the bicycle cannot perfectly follow the formula, 
and there will always be some minor changes in speed. Moreover, the measurement 
model, which translates an estimated state into a measurement, will also 
contain some uncertainty since the camera does not have a perfect 
representation of the space.

We can now describe these models formally. Note that since the state is a 
vector, we will use $\mathbf{x}$ for vectors. The motion and measurement models 
can be generally expressed as:

\begin{align*}
    p(\vecat{x}{k} | \vecat{x}{k-1}) 
        &= f_t(\vecat{x}{k-1}, \vecat{u}{k}, \vecat{w}{k}), \\
    p(\vecat{z}{k} | \vecat{x}{k}) 
        &= h_t(\vecat{x}{k}, \vecat{v}{k}),
\end{align*}

where $f_t$ and $h_t$ are functions, $\vecat{x}{k-1}$ is the estimated state 
from time step $k-1$, $\vecat{u}{k}$ is called the input (or control) vector, 
and $\vecat{w}{k}$ and $\vecat{v}{k}$ are zero-mean noise random variables.

However, this definition is too general since the functions $f_t$ and $h_t$ can 
be anything. As mentioned earlier, to get a closed-form solution in the 
Bayesian inference framework, we need to choose conjugate distributions for the 
likelihood and the prior. The Kalman filter, which we will describe soon, works 
for the Gaussian-linear case, where the functions $f_t$ and $h_t$ are linear 
and noise variables are distributed as Gaussian, i.e., 
$\vecat{w}{k}, \vecat{v}{k} \sim \mathscr{N}(\mathbf{0}, \Sigma)$. The Gaussian-
linear state-space model has the following representation:

\begin{align}
    p(\vecat{x}{k} | \vecat{x}{k-1}) 
        &= \mathbf{F} \vecat{x}{k-1}
            + \mathbf{B} \vecat{u}{k}
            + \vecat{w}{k},
        & \vecat{v}{k} \sim \mathscr{N}(\mathbf{0}, \mathbf{Q}), \\
    p(\vecat{z}{k} | \vecat{x}{k})
        &= \mathbf{H} \vecat{x}{k} + \vecat{v}{k},
        & \vecat{w}{k} \sim \mathscr{N}(\mathbf{0}, \mathbf{R}), \\
    p(\vecat{x}{0}) &\sim \mathscr{N}(\vecat{\hat{x}}{0}), \vecat{P}{0}),&
\end{align}

where:

\begin{itemize}
    \item $\mathbf{F}$ is the \textit{transition matrix},
    \item $\mathbf{B}$ is the \textit{input matrix},
    \item $\mathbf{H}$ is the \textit{measurement matrix},
    \item $\mathbf{Q}$ and $\mathbf{R}$ are symmetric positive definite matrices
        that describe the statistical properties of the motion noise
        \vecat{v}{k} and the measurement noise \vecat{w}{k}, respectively,
    \item $\vecat{\hat x}{0}$ and $\vecat{P}{0}$ are mean and variance of the 
    prior state.
\end{itemize}

The control variable $\vecat{u}{k}$, in general, represents some input signal from the environment, and $\mathbf{B}$ specifies how the input signal affects the dynamic system. This may include the effect of gravity on the vertical traveled distance or the voltage applied to some circuit. However, in object tracking, we rely on measurements obtained from sensors to update our estimate of the object's state. We do not have control over the object's motion. Thus, the input matrix is often redundant, and we omit it (or set it to a zero matrix) along with the input variable $\vecat{u}{k}$.

The notation above may be inconvenient in some cases. In this work, we will often use a shorter version that has the same meaning:

\begin{align}
    p(\vecat{x}{k} | \vecat{x}{k-1})
        &= \mathscr{N}(\vecat{x}{k}; \mathbf{F}\vecat{x}{k-1}, \mathbf{Q}), 
        \label{eq:motion-model} \\
    p(\vecat{z}{k} | \vecat{x}{k})
        &= \mathscr{N}(\vecat{z}{k}; \mathbf{H}\vecat{x}{k}, \mathbf{R}), 
        \label{eq:measurement-model} \\
    \vecat{x}{0}
        &\sim \mathscr{N}(\vecat{x}{0}; \vecat{\hat{x}}{0}, \vecat{P}{0}).
        \label{eq:prior-x0}
\end{align}

\subsubsection{Constant Velocity Model}\label{sec:cv-model}

The Constant Velocity (CV) model is one of the most commonly used motion and measurement models in the context of object tracking. It describes the kinematics of objects in a 2D space that move with a constant velocity. The state vector comprises a position vector and a velocity vector, and while the position vector contains the coordinates of the object in the space, the velocity vector contains the object's speed in the direction of each axis.

The CV model is linear and one of the simplest models. It assumes that the speed of objects remains unchanged during tracking, with only small deviations from the constant. This model can be reasonably utilized in scenarios where objects do not change their speed or direction, such as vehicles on highways or planes in the sky. Since the state vector contains only four variables, the use of this model does not increase the computational burden on tracking algorithms.

The motion equation for the CV model is given by \ref{eq:motion-model}, and the state vector $\vecat{x}{k}$, the state transition matrix $\mathbf{F}$, and the process noise matrix $\mathbf{Q}$ can be defined as:

\begin{equation}
    \vecat{x}{k} =
    \begin{bmatrix}
        x_{1,k} \\ 
        x_{2,k} \\ 
        v_{x_1,k} \\ 
        v_{x_2,k}
    \end{bmatrix};
    \quad
    \mathbf{F} =
    \begin{bmatrix}
       1 & 0 & dt & 0 \\
       0 & 1 & 0 & dt \\
       0 & 0 & 1 &  0 \\
       0 & 0 & 0 &  1 
    \end{bmatrix};
    \quad
    \mathbf{Q} = q^2 \cdot
    \begin{bmatrix}
        \frac{dt^3}{3}    & 0                 & \frac{dt^{2}}{2}  & 0  \\
        0                 & \frac{dt^3}{3}    & 0                 & \frac{dt^{2}}{2} \\
        \frac{dt^{2}}{2}  & 0                 & dt                & 0 \\
        0                 & \frac{dt^{2}}{2}  & 0                 & dt
    \end{bmatrix},
\end{equation}

where $dt$ is the change in time between the last estimation and the newly computed one, and $q$ is the motion noise parameter, which represents the uncertainty in the state transition.

The measurement model transforms a state vector from the state space into the measurement space. Since the vanilla Kalman filter, which we will derive soon, works only with linear models, the measurement model for the CV model assumes sensors measurements in the same 2D space as in the state vectors. The measurement equation is given by \ref{eq:measurement-model}, and the measurement matrix $\mathbf{H}$ and the measurement noise matrix $\mathbf{R}$ can be defined as:

\begin{equation}
    H =
    \begin{bmatrix}
        1 & 0 &0 & 0 \\
        0 & 1 &0 & 0
    \end{bmatrix};
    \quad
    R =
    r^{2}\cdot
    \begin{bmatrix}
        1 & 0 \\
        0 & 1
    \end{bmatrix},
\end{equation}

where $r$ is the measurement noise parameter, which determines the variance of the measurement noise.

We need to describe the choice of parameters $q$ and $r$ in more detail. These parameters are crucial in obtaining good estimates from a filter. These parameters are not known a priori and their values should be chosen carefully by the means of trial and error, or using automated methods like, for instance, in \cite{bulutProcessMeasurementNoise2011}. However, selecting appropriate values often involves a trade-off between tracking accuracy and computational complexity, and the use of exact methods depends on the application and one's requirements.

The CV model can be extended with additional information, such as change of speed. In this way we will come to the constant acceleration model. However, it will not be used in this work and, therefore, we will not leave formal definition of this model here.
