As we have presented the results of performance measurement, let us discuss them and draw several conclusions. In general, the GM-PHD filter is a robust algorithm. However, like any other algorithm, it has its drawbacks.

First of all, it is obvious that the filter is sensitive to the quality of data coming from a sensor. For instance, if the sensor has a low detection probability and high clutter rates, tracking performance may suffer from the uncertainty on both the state estimate precision and the estimated number of targets. Moreover, the filter becomes sensitive to the setting of the survival probability. For these cases, it is recommended to either use additional sources of information or use another tracking algorithm.

Secondly, let us discuss track maintenance. Generally, the GM-PHD filter has an in-built implicit track maintenance. When a Gaussian has a sufficient weight, in our case greater than $0.5$, the track is considered initialized and confirmed. However, there is a problem. If several clutter measurements happen to be in the vicinity of birth components, the likelihood of these components getting enough weights is quite high. This will cause tracks with only a few estimates, often only one. This problem may be addressed by increasing the minimum weight required for generating an estimate or by introducing some additional logic, such as the minimum number of estimates, something like the M/N logic discussed in Section \ref{sec:track-maintenance}.

The next problem is the consequence of the simple strategy we have chosen for tagging Gaussian components to connect estimates into trajectories. We have seen that it suffers from the track continuity problem, which is discussed in detail along with the reason in Section \ref{sec:c2-results}. We may try to address this issue by introducing more sophisticated techniques of track maintenance. For instance, we could track the history of track estimates for each tag, and if the correct tag would have been discarded in favor of a tag with an empty history, we would detect it and assign the correct tag to preserve the track continuity. The other option would be to use clustering techniques like in the JPDA filter.

Next, we have observed the change of performance when there are no fusion components. They are fundamentally similar to birth components, but we have shown how they could be used for inputting additional external information into the filter. For blind zones, there is no way to append information using the standard techniques. The filter is unable to capture targets that are invisible to a sensor, and the way how the posterior intensity is represented in the GM-PHD filter makes it impossible to confirm tracks since there are no Gaussian terms to which new measurements can be assigned. We could have defined the birth components that cover the whole field of view, however, it would lead to the problem discussed in the previous paragraph, i.e., too many false estimates because of confirming birth components with clutter measurements.

Last but not least, it is obvious that the complexity $\mathcal{O}(mn)$ discussed in Section \ref{sec:phd-filter} assumes that the speed of the GM-PHD filter is highly dependent on the clutter rate. If there are too many false positive measurements at each time step, the performance in terms of processing time degrades. This is another reason why it is required to reduce the clutter rate as much as possible.

Despite all the drawbacks we have discussed, the GM-PHD filter remains one of the most popular and robust solutions to multi-target tracking. The filter is fast and reliable; it does not need to estimate the full posterior distribution, which is often complex. For Gaussian-linear cases, the GM-PHD filter has a closed form solution, and non-linear models can be estimated using the UK-PHD or the EK-PHD filter.
