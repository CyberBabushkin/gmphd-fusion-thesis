The thesis is structured as follows. Chapter 1 provides the theoretical background necessary for understanding the rest of the thesis, including the introduction of probability theory and probability distributions and the Bayes' rule. Next, we present the Bayesian inference framework, which is the basis for the general Bayes filter, a recursive algorithm for state estimation. Finally, we present the concepts of state-space models and the Kalman filter.

Chapter 2 is dedicated to multi-target tracking. It starts with an overview of target tracking methods, followed by an introduction to Random Finite Sets (RFS). The chapter then describes the PHD filter and the PHD function. Next, we give a PHD filter formal definition, and then observe the Gaussian Mixture PHD filter. The chapter concludes with a discussion of track maintenance, the GM-PHD recursion, and ends with the framework for external information fusion.

Chapter 3 describes the implementation and testing methodology of the GM-PHD filter. We then introduce several metrics that are used to measure the performance of multi-target tracking algorithms, then we describe four testing scenarios. The final section on parameters testing and methodology explains how the tests were conducted and how the results were analyzed.

Chapter 4 analyzes the results of the tests conducted in Chapter 3. This chapter includes sections on test results for each scenario and a discussion of the results, where a comprehensive analysis of the filter's strengths and weaknesses is provided.

The final Chapter 5 concludes this work by summarizing the main contributions of the thesis and highlighting areas for future research.
