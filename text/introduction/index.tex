Object tracking is a critical problem in signal processing and computer vision. The primary goal of tracking algorithms is to estimate the states of moving targets based on a sequence of sensor measurements in environments with a high degree of uncertainty. Different domains of the problem require different approaches to target tracking. In this work, we provide an overview of the existing areas of target tracking and introduce the scope of this study.

Tracking algorithms can be categorized based on the number of targets they track, whether it is a fixed or known number of targets, or a variable and unknown number of targets. When the number of targets in the tracking area is greater than one, we refer to it as multi-target tracking (MTT). Object tracking is widely used in various fields, including robotics, autonomous vehicles, surveillance, and medical imaging. With the recent emergence of deep learning algorithms, there has been renewed interest in MTT, and many new deep learning-based approaches have been proposed. However, the learned model predictive power is questionable, as these approaches estimate the internal object dynamics using complex non-linear models, while the Bayesian approach assumes fixed known model that are specified manually according to some prior knowledge, for instance, physical laws. In this study, we focus on MTT using Bayesian inference, the statistical approach based on recursive updates of the target posterior distribution using a set of measurements.

Bayesian algorithms are classified based on the results they achieve \cite[11]{sarkkaBayesianFilteringSmoothing2013}. The first type of algorithms are the Bayesian smoothers, which remove the noise of states based on past and future measurements, i.e., they smooth the signal. The other type are the Bayesian predictors, which predict the future state of a target more than one step ahead based on past measurements. The last type are the Bayesian filters, which predict the current state of a target based on measurements up to the moment of the predicted state. The Kalman filter, the fundamental algorithm for many modern and sophisticated tracking methods, is an example of Bayesian filters. This study focuses on Bayesian filters, and we cover the Kalman filter in detail.

Algorithms can also be separated based on the type of sensor and measurements used. There are several sets of algorithms available to address specific problems depending on the sensors used. For instance, point object tracking is used when the sensor generates one measurement per object, such as a radar or a camera with an object recognition algorithm. When lidars are used, one target generates a set of measurements, and this is referred to as extended object tracking \cite{granstromExtendedObjectTracking2017}. In surveillance, especially when tracking people, group object tracking algorithms are used to estimate groups of individuals rather than every individual separately \cite{salmondGroupExtendedObject1999}. Additionally, tracking with multipath propagation refers to scenarios where a sensor receives more than one detection caused by the multipath phenomenon or multiple reflections from surrounding objects of the same signal \cite{bar-shalomTrackingLowElevation1994}. Finally, tracking with unresolved targets occurs when several targets produce only one measurement \cite{angleMultipleTargetTracking2021}. In this study, we focus on point object tracking, assuming that one target produces at most one measurement, and each measurement can be generated by at most one target.

This study aims to enhance the performance of the Gaussian Mixture probability hypothesis density (GM-PHD) filter, a popular Bayesian filter for multi-target tracking (MTT), by proposing a novel approach to incorporate external information. Specifically, we provide a detailed description of the GM-PHD filter and its internals, and demonstrate through an example how the filter may fail under environmental limitations. We then introduce our proposed method for integrating external information and analyze its impact on the filter's performance. Finally, we evaluate the filter's strengths and weaknesses in various scenarios and discuss the significance of our contribution in improving the accuracy and robustness of the GM-PHD filter for MTT.
