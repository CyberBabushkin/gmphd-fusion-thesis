Tato práce se zaměřuje na problém sledování více cílů v prostředí s vysokým množstvím rušení a nejistotou ohledně počtu sledovaných objektů pomocí Bayesovské inference. V této práci se hlavně zaměřujeme na populární Gaussian mixture probability hypothesis density (GM-PHD) filtr a představujeme techniku, jak zahrnout dodatečné informace pro případy, kdy senzor nedokáže detekovat sledované objekty z důvodu fyzických nebo prostředkových omezení. Poskytujeme veškeré potřebné teoretické pozadí od základů teorie pravděpodobnosti až po diskusi o různých metodách sledování více cílů a představení rámce Finite Set Statistics (FISST). Dále zahrnujeme rozsáhlé měření výkonnosti a analýzu výsledků, kde ukazujeme, že navržená technika fúze výrazně zlepšuje sledovací výsledky. Nakonec diskutujeme omezení filtru a navrhujeme možné způsoby, jak je překonat.
