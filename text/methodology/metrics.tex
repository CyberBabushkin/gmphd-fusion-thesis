For evaluating the performance of the GM-PHD filter, we used two metrics proposed in \cite{voGaussianMixtureProbability2006}. The first metric is called the Circular Position Error Probability (CPEP). It measures the probability that the estimated position of a target falls in a circular region with a given radius around the true position. This metric has its roots in the military area, where the weapons system's precision is measured with the probability of hitting targets in a certain ellipse with a given mean \cite{nelsonUseCircularError}. While linear metrics like a simple Euclidean distance would measure the accuracy, the CPEP metrics measures precision of the tracking algorithm. In formal language, this metric is defined as:

\begin{equation}
    \operatorname{CPEP}_k (r) = \frac{1}{|X_k|} \sum_{\mathbf{x} \in X_k} \rho_k(\mathbf{x}, r),
\end{equation}

\noindent where $r$ is the radius of the circluar area, and $\rho_k(\mathbf{x}, r)$ is defined as:

\begin{equation}
     \rho_k(\mathbf{x}, r) = \Pr{\| \mathbf{H}_k \mathbf{\hat{x}} - \mathbf{H}_k \mathbf{x}\|_2 > r, \forall \mathbf{\hat{x}} \in \Hat{X}_k}.
\end{equation}

\noindent Here, $\hat{X}_k$ refers to the set of estimates generated by the tracking algorithm at time step $k$, $\mathbf{H}_k$ is the single-target measurement model, and $\| \cdot \|_2$ is the Euclidean norm of a vector.

The second measure of tracking error is the expected absolute error on the number of targets. In simple words, this metric gives the average difference between the predicted and true number of targets. Mathematically, it is defined as:

\begin{equation}
    E\left[ \left||\hat{X}_k| - |X_k|\right| \right].
\end{equation}

Generally, these two metrics measure the total uncertainty of a multi-target tracking filter given be a random finite set -- both the number of targets in the set and the precision of the estimates. We evaluate these metrics for all tracking scenarios to show the influence of a parameter change or the presence of fusion techniques on the filter's performance.
